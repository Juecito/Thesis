
% this file is called up by thesis.tex
% content in this file will be fed into the main document
%----------------------- introduction file header -----------------------
%%%%%%%%%%%%%%%%%%%%%%%%%%%%%%%%%%%%%%%%%%%%%%%%%%%%%%%%%%%%%%%%%%%%%%%%%
%  Capítulo 1: Introducción- DEFINIR OBJETIVOS DE LA TESIS              %
%%%%%%%%%%%%%%%%%%%%%%%%%%%%%%%%%%%%%%%%%%%%%%%%%%%%%%%%%%%%%%%%%%%%%%%%%

\chapter{\textcolor{Azul}{Introducción}}

%: ----------------------- HELP: latex document organisation
% the commands below help you to subdivide and organise your thesis
%    \chapter{}       = level 1, top level
%    \section{}       = level 2
%    \subsection{}    = level 3
%    \subsubsection{} = level 4
%%%%%%%%%%%%%%%%%%%%%%%%%%%%%%%%%%%%%%%%%%%%%%%%%%%%%%%%%%%%%%%%%%%%%%%%%
%                           Presentación                                %
%%%%%%%%%%%%%%%%%%%%%%%%%%%%%%%%%%%%%%%%%%%%%%%%%%%%%%%%%%%%%%%%%%%%%%%%%

\section{Presentación} % section headings are printed smaller than chapter names

\textcolor{Azul}La primera concepción de la palabra robot proviene de la obra \textit{Rossum's Universal Robots} del novelista checo Karel Capek[1]. En la lengua checha \textit{robota} significa trabajo, y la RAE [2] define a un robot como \textit{Máquina o ingenio electrónico programable, capaz de manipular objetos y realizar operaciones antes reservadas solo a las personas}. \\\\En la industria, los robots  más utilizados son conocidos como robots manipuladores, cuya estructura se asemeja a los brazos humanos, Fig. (\ref{estructurarobot}) [3]. Se estructura está formada por eslabones unidos mediante articulaciones para brindar libertad de movimiento en el entorno de trabajo, donde uno de los extremos del manipulador permanece fijo mientras que el otro extremo cuenta con una herramienta que le permite realizar una tarea para la que fue programado. Los robots manipuladores son utilizados en gran parte de las tareas como seleccionar y distribuir objetos, o para realizar modificaciones en su entorno como pintar o ensamblar, entre muchas otras, reduciendo los tiempos de operación y minimizando la necesidad de supervisión.\\\\Un robot manipulador se constituye de cuatrp sistemas fundamentales. El primero es un sistema mecánico que le da forma al cuerpo del robot y está constituido pricipalmente por eslabones y articulaciones que permiten el movimiento en su espacio de trabajo, aunque también forman parte del sistema mecánico elementos de transmisión de potencia, como engranes, cadenas y poleas; elementos que otorgan estabilidad al robot, como los sistemas de contrapesos, ya sean fijos o móviles; y elementos de unión, como soldadura y tornillos. El segundo es un sistema eléctrico-electrónico que otorga energía para el movimiento de los motores, para la activación de sensores y la distribución de señales. El tercer sistema pertenece a la programación, que constituye la comunicación entre el humano y la máquina. El cuarto  sistema es de control, que describe de manera matemática al robot y permite su manipulación para la realización de una tarea asignada.

\begin{figure}
	\centering
	\includegraphics[scale=0.5]{Capitulo1/figs/EstructuraManipulador.png} 
	\caption{Manipulador robótico y sus partes equivalentes en el cuerpo humano}
	\label{estructurarobot}
\end{figure}



%%%%%%%%%%%%%%%%%%%%%%%%%%%%%%%%%%%%%%%%%%%%%%%%%%%%%%%%%%%%%%%%%%%%%%%%%
%                           Objetivo                                    %
%%%%%%%%%%%%%%%%%%%%%%%%%%%%%%%%%%%%%%%%%%%%%%%%%%%%%%%%%%%%%%%%%%%%%%%%%

\subsection{Objetivo}

Diseño y manufactura de un robot manipulador con 5 grados de libertad capaz de realizar seguimiento de trayectorias.
%%%%%%%%%%%%%%%%%%%%%%%%%%%%%%%%%%%%%%%%%%%%%%%%%%%%%%%%%%%%%%%%%%%%%%%%%
%                           Motivación y estado del arte                %
%%%%%%%%%%%%%%%%%%%%%%%%%%%%%%%%%%%%%%%%%%%%%%%%%%%%%%%%%%%%%%%%%%%%%%%%%
\subsection{Motivación}

Los robots manipuladores tienen un amplio uso en la industria y en áreas donde se pretende replicar el trabajo que realiza una persona, ya sea porque es demasiado complejo o se trata de procesos donde se requiera una gran precisión, por encontrarse en ambientes peligrosos o tóxicos, etc. Sin embargo, los robots manipuladores también pueden ser de gran utilidad en el ámbito educativo. Los estudiantes de robótica suelen tener acceso nulo o restringido a manipuladores ya sea por su delicadeza o por la intención de no dañar al robot debido a los elevados costos de reparación. La creación de un manipulador con 5 grados de libertad a un bajo costo pondrá al alcance de los estudiantes una herramienta práctica en el área de la robótica industrial.
%%%%%%%%%%%%%%%%%%%%%%%%%%%%%%%%%%%%%%%%%%%%%%%%%%%%%%%%%%%%%%%%%%%%%%%%%
%                   Planteamiento del problema                          %
%%%%%%%%%%%%%%%%%%%%%%%%%%%%%%%%%%%%%%%%%%%%%%%%%%%%%%%%%%%%%%%%%%%%%%%%%

\subsection{Planteamiento del problema}

El diseño de robots manipuladores es una temática ampliamente abarcada en los trabajos escolares de nivel licenciatura, aunque la mayor parte de esa literatura tiene por objetivo demostrar conceptualmente las capacidades de los robots manipuladores. La construcción del prototipo funcional es un elemento que no se suele llevar a término.
%%%%%%%%%%%%%%%%%%%%%%%%%%%%%%%%%%%%%%%%%%%%%%%%%%%%%%%%%%%%%%%%%%%%%%%%%
%                           Metodología                                 %
%%%%%%%%%%%%%%%%%%%%%%%%%%%%%%%%%%%%%%%%%%%%%%%%%%%%%%%%%%%%%%%%%%%%%%%%%
\subsection{Metodología}

El proceso de construcción del manipulador comienza con el diseño, para ello se plantean los requerimientos y se traducen a especificaciones del robot. Posteriormente se elabora un diseño asistido por computadora donde se obtiene la geometría y espacio de trabajo del manipulador. Se realiza la descripción analítica del movimiento mediante la cinemática directa e inversa, y se termina con la construcción de las piezas y el ensamble del prototipo.

%%%%%%%%%%%%%%%%%%%%%%%%%%%%%%%%%%%%%%%%%%%%%%%%%%%%%%%%%%%%%%%%%%%%%%%%%
%                         Contribuciones                                %
%%%%%%%%%%%%%%%%%%%%%%%%%%%%%%%%%%%%%%%%%%%%%%%%%%%%%%%%%%%%%%%%%%%%%%%%%

\subsection{Contribuciones}

La principal contribución de este trabajo es la simplificación del proceso de construcción para un robot manipulador con 5 grados de libertad, con los elementos necesarios para ser replicado en posteriores trabajos y reduciendo los costos asociados a la fabricación.

%%%%%%%%%%%%%%%%%%%%%%%%%%%%%%%%%%%%%%%%%%%%%%%%%%%%%%%%%%%%%%%%%%%%%%%%%
%                           Estructura de la tesis                      %
%%%%%%%%%%%%%%%%%%%%%%%%%%%%%%%%%%%%%%%%%%%%%%%%%%%%%%%%%%%%%%%%%%%%%%%%%

\subsection{Estructura de la tesis}

En este capítulo se hace una breve descripción del trabajo a desarrollar, el objetivo y algunas ideas generales de robots manipuladores, en el siguiente capítulo se encuentran los fundamentos teóricos que sirven para entender las generalidades de los manipuladores, su composición y la forma de operarlos. En el tercer capítulo se aborda el diseño mecánico junto con un análisis de los elementos que conforman al manipulador, las geometrías y los espacios de trabajo que definirán la movilidad. En el capítulo cuatro se lleva a cabo la construcción del prototipo y la puesta en marcha. En el quinto capítulo se analizan los resultados obtenidos donde se observa la capacidad real del robot manipulador construido y en el último capítulo se tratan las conclusiones y el posible trabajo a futuro del proyecto.
\\

\section{Marco teórico}

Escribo unas lineas por acá

\paragraph{algoporaqui}

Escribo otras lineas por acá

